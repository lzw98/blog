% !TeX root = ../XMU.tex
\chapter{引言}{Introduction}

%以下所述内容基本来自于 徐延的基于数据挖掘的财务。的研究,在查重前需要不断进行修改


中国的股票市场自1989年成立以来历经40年时间。在此期间,上市公司数量和总体规模 增长迅速,为我国经济发展注入了强劲动力。但是由于我国证券市场发展时间相对较短,相关监管政策不健全等原因,上市公司财务造假仍呈现高发态势,损害了投资者的切身利益, 影响了中国证券市场的健康稳定发展。因此,如何对上市公司财务造假问题进行精准识别和 有效预警就成为了监管层、机构投资者和个人投资者共同关注的问题。


财务造假一直是与财务报告相伴相生的一个问题,也是困扰监管层、金融业界和投资者多年的难题。某些企业出于粉饰公司业绩、提升投资者信心或逃避监管等目的,通过虚增交易、隐瞒相关信息、篡改财务数据等手段进行财务造假。常见手段包括特殊交易的不当核算(如对债权、债务重组以及关联交易的错误核算)、错用会计准则以及虚构交易、资产负债造假等等。这种行为不仅影响了投资者对公司情况的正确判断,还挑战了法律监管的权威。财务造假的“爆雷效应”还容易导致重大金融风险。因此,如何对上市公司财务造假问题进行精准识别和有效预警成为了各方面共同关注的问题。


回顾国外证券市场的发展史,我们可以发现,财务造假的先例可以追溯至英国南海公司的财务造假案,这起案件也被认为是非官方审计的开端。近几十年来,即使在监管体系较为完善的发达国家资本市场,上市公司财务造假案件也时有发生。其中,美国Enron公司\cite{noauthor_2002:_nodate}财务造假案影响较为巨大。


与发达国家相比,我国的证券业务开展时间较晚,部分上市公司抱有侥幸心理,各种违规现象时有发生。从违规类型来看,其中相当一部分属于财务造假。其中不仅有蓝田股份等典型案例,也有北大荒、雅百特等近年来新出现的案例。


对上市公司财务造假案例进行分析后,有研究者认为,只要公司的内部控制失效,公司管理层凌驾于内控体系之上,任何公司都有发生财务舞弊的可能,只是财务舞弊的手段和隐蔽性各有不同。


% 总的来看,从事高风险业务(农业、制造业、能源行业等)、拥有高风险财务结构的公司是财务舞弊现象的高发领域,被审计机构出具保留意见报告甚至不发表意见是财务造假的强烈信号。从造假动机来看,公司财务造假主要分为两种情况。在公司准备上市的阶段,出现财务造假主要体现在公司IPO的申请材料中,通过财务造假以满足《证券法》中关于公司上市的条件。公司上市后,出现财务造假的主要目的是美化公司业绩、提升投资者信心,以抬升股价,或为了避免退市等。因为公司上市前的数据获取存在较大难度,本文主要研究对象为公司己经成功上市后,处于上述目的而进行的财务造假行为。从造假手段来看,最常见的就是虚增交易。上市公司通过伪造单据、伪造合同虚构原本不存在的交易事项,以此形成虚假的收入和利润,这种方法可以快速改善公司财务报表中的收入和利润,形成公司发展态势较好的假象。另外,还有的公司通过虚增资产的方式进行财务造假,具体表现在对某些资产的估值严重偏离公允价值,以此达到虚增公司资产的目的。还有的公司通过提前确认收入的方式虚增收入,或者隐瞒重大事项等方式进行财务造假。特别是关联公司之间的交易问题是重灾区,控股公司和公司之间的关联交易常常偏离公允价格,以实现操纵上市公司利润的目的。从处罚力度来看,发达国家对上市公司财务造假的打击力度较大,不仅公司受到处罚,董事会成员和高管也会面临来自法院的刑事指控。另外,相关的会计师事务所和投行也会面临监管部门和司法部门的严厉处罚,总的来看违法的成本较高。在我国,相关法律法规对上市公司财务造假的处罚相比西方国家要偏轻,对相关责任人继续从业的限制规定时间也较短,导致一些公司仍然冒着违法风险进行财务造假。

本文尝试构建一个公司财务造假风险识别模型,为财务造假的识别提供技术支持和一定辅助。从资本市场及其参与者角度来看,本文的研究有以下几方面的现实意义。

首先,促进上市公司规范经营。加强对上市公司财务造假的识别,将提升上市公司的违法成本,可以促进上市公司和规范经营和真实披露,通过规范经营来获取经营业绩,而非通过财务造假来获得有吸引力的财务数据。

第二,提高资本市场效率。上市公司财务造假可能导致资本市场信息不真实,降低资本市场效率。因此加强对上市公司财务造假的识别有助于真实准确反映企业经营状况,提升资本市场效率,促进资本市场健康稳定发展。

第三,强化监管能力。以往的财务造假识别往往基于审计人员的主观判断,主要来自于审计人员对三张财务报表之间的逻辑关系进行推断分析,找出矛盾点和可疑点,从而对财务造假行为进行识别。但是这种识别方法也存在着一些不足,如耗时长、成本高、错误率和遗漏率高、过度依赖个人经验、缺乏统一标准等问题。因此,引入效率和准确率更高的算法和设计对于财务造假的精准识别具有重要意义。本文对上市公司财务造假识别的研究,将有助于监管部门加强对上市公司财务造假的预警和监测,提升违法成本,促使上市公司如实反映经营业绩。

第四,保护上市公司和投资者利益。在市场参与者层面,研究财务造假识别对维护上市公司和投资者利益都具有重要意义。财务数据是投资者进行投资决策的主要依据之一,篡改后的财务数据往往使得公司显得更有发展前景,对于证券投资者行为会产生误导,最终导致投资损失。对于上市公司而言,如果财务造假行为得不到有效遏制,就会在市场中引起“劣币驱逐良币”效应,损害上市公司利益。因此加强对上市公司财务造假识别的研究实质上有助于保护市场参与者的切身利益。