% !TeX root = ../XMU.tex
\chapter*{摘要}
%不要用我,我们,什么什么是当下的热点问题
%用本文,可以用被动语态
%科学的描述,引入要快,不用太长
%不要自问自答
%研究什么,用了什么方法研究,得出什么结论
%十句话左右

%presentation不要按论文的格式定
%建议直接抛出什么问题

由于我国证券市场发展时间相对较短,相关监管政策不健全等原因,上市公司财务造假仍呈现高发态势。为实现对财务欺诈行为进行准确预测,本文基于2018年上市公司的财务数据,构建了logistic模型和随机森林模型作为预测模型。logistic模型研究发现,中国的上市公司往往会选择通过对营业成本进行改动以对财务报表进行粉饰;在偿债能力越强的公司,其财务主管领取薪酬会更加减少财务造假的可能。随机森林模型的研究发现,利润总额增长率和可持续增长率是判别一个公司是否产生财务造假重要指标。

本文的研究为中国上市公司的财务造假识别提供了新的角度与方法,建立的模型有利于资源配置优化。为审计师审计财务造假提供了新的参考依据。	

\keywords{财务造假;logisitc模型;随机森林模型;营业成本;偿债能力;利润总额增长率;可持续增长率}	



\chapter*{\bfseries Abstract}

Due to the relatively short development time of China's securities market and the inadequacy of relevant regulatory policies, financial fraud in listed companies is still on the rise. In order to predict financial fraud accurately, this paper builds a logistic model and a RandomForest model based on the financial data of listed companies in 2018. 
The logistic model research found that Chinese listed companies often choose to decorate their financial statements by changing operating costs. In companies with stronger solvency, their financial officers will be more likely to receive financial remuneration when they receive compensation. The study of the Random orest model found that the total profit growth rate and sustainable growth rate are important indicators for judging whether a company has financial fraud.

The research in this paper provides a new perspective and method for the identification of financial fraud in Chinese listed companies. The model established is conducive to the optimization of resource allocation. It provides a new reference for auditors to audit financial fraud.

\englishkeywords{financial fraud; logisitc model; RandomForest model; operating cost; debt service ability; total profit growth rate; sustainable growth rate}

