% !TeX root = ../XMU.tex
\chapter{文献综述}{Literature Review}

%需要讲出自己的论文有什么贡献,highlight your study's contribution
%一定要好好调研有没有人做过一样的
%ABC考虑了,但有一些问题没有透
%DEF考虑了,但是没有考虑透
%我考虑了,而且有意义
%不但要自己写文献综合,还要讲其优缺点
%用实证的角度来说明最新的造假手段

\section{财务造假原因综述}{Summary of Financial Fraud Reasons}

关于财务造假的成因,国外研究者提出了三角理论、GONE成因理论等。\citep{cresscy1973other}提出著名了财务造假三角理论,企业进行财务造假的诱因主要由三个方面,即压力、机会和自我接受。压力即代表了目前公司遇到问题的紧迫性,如利润不达标即将暂停上市、公司收入降低影响股价、公司现金流紧张等,在这种情况下,公司进行财务造假就有了行为上的动机。

%机会即公司内部管理制度的缺失,给财务造假行为提供了客观条件。自我接受(也称自我合理化)是指公司财务造假所使用的借口,如帮助公司暂时度过难关、维护股民利益等,这是公司管理层策划财务造假时的主管心态。这三个诱因一个是动机,一个是条件,一个是目的,少了任何一项,都无法支撑企业的财务造假行为。

\citep{bologna1993accountant}提出GONE\footnote{GONE由四个单词:greed,opportunity,need, exposure首字母组成}成因理论,也被称作“四因子理论”.该理论认为,公司财务造假的原因主要可以分为贪婪、机会、需要和暴露四个因子组成。此理论把个体的主观因素归为贪婪和需要,阐述了造假者的动机;机会和暴露更多是指财务造假的客观条件,即内控机制不完善带来的造假机会和造假行为被发现和惩罚的可能性。这几类因素从不同的层面影响公司的决策行为,最终造成了财务造假行为的产生。企业董事会或管理层有不良动机,有粉饰公司经营状况现实需要,且公司客观条件可以进行财务造假,且事后不易被发现,那么企业就可能对公司的经营情况进行虚假的说明。

在GONE理论提出后,后续的研究不断对其进行完善。如财务造假因子理论把GONE理论中的成因分为主观和客观,主观条件是指个人的道德品质、造假动机等,客观条件是指公司内控严格程度、被发现的可能性以及惩罚力度等。两类因素共同作用,决定了一个公司是否会进行财务造假。

在国外理论研究较为完善的基础上,近年来国内学者对上市公司财务造假也进行了研究。如\citep{Mei}以2006年至2015年因财务造假受到监管机构处罚的上市公司为样本进行研究,发现公司财务造假的成因可以是为了成功进行IPO或取得增发资格、对投资者隐瞒经营状况、防止出现ST、操纵股价等。


\section{财务造假识别研究综述}{Review of Research on Financial Fraud Identification}

财务造假的识别研究一直与财务造假本身的研究是相伴相生的,数十年来,国内外研究者己经对此问题进行了深入的研究和讨论,取得了一系列研究成果。相关研究主要分为三个阶段。

第一阶段是案例分析和特征总结,研究者主要对上市公司财务造假的典型案例进行分析比较,归纳总结这些公司及其财务数据多一些特征和矛盾点,研究变量和研究方法较为单一。\citep{pavlovic_fraud_2019}将Benford定律\footnote{由统计学家Benford于1938年提出,得出人们在处理的数据中,1-9作为首位数出现的概率是不同的}应用于财务造假的研究,利用塞尔维亚工商局2008年-2013年的数据,说明财务数据中,若第二个数字明显偏离Benford定律,那么有可能是公司对财务数据进行了改动。

第二阶段是建模分析,研究者开始由定性研究转为定量研究,通过建立数学模型对公司财务数据进行分析研判,从中寻找财务造假的痕迹。如\citep{md_nasir_real_2018}使用SCM(Securities Commissions of Malaysia)和Bursa Malaysia数据库对马来西亚上市公司的2001年到2008年的财务造假行为建立时间序列预测模型,进行实证分析。研究结果表明,在财务造假发生的前四年,盈余管理往往是非常有效的;生产成本在财务造假公司与非财务造假的公司的显著不同往往只发生在财务造假发生的前两年;Malaysia的上市公司往往喜欢通过操纵应计费用项目来进行财务造假等;\citep{gao_go_2017}通过实证研究表明财务造假的过程中往往伴随着外部董事的异常换手率,而且在此期间的换手率还同时与上市板块,会议频繁程度,以及财务高管的数量有关,研究还发现财务问题较多的公司其公司高管的离职率也会越高。在国内,\citep{liao_corporate_2019}实证研究表明,中国的企业社会责任可能是减少公司欺诈发生的有效途径。

第三阶段是大数据运用。随着信息技术的发展研究者开始使用数据挖掘技术对公司财务造假问题进行研究,取得了一定的研究成果。\citep{tseng_quadratic_2005}使用logistic模型研究财务报表风险甄别,得出logistic模型优于判别分析模型;\citep{chen_enhancement_2017}建立SVM模型检测财务造假,结果表明SVM模型的预测效果较好;\citep{abbasi_metafraud:_2012}使用在包含数千家合法和欺诈性公司的数据上进行了一系列实验,得出其使用的meta-learning(元学习)框架在对于财务欺诈具有较好的效果。结果表明,框架的每个组成部分均对其整体有效性做出了重要贡献,额外的实验证明了元学习框架相对于最新的欺诈检测方法的有效性。此外,该框架会生成与每个预测相关的置信度分数,这可以前所未有地促进财务欺诈检测性能,并可以作为有用的决策辅助工具。国内使用数据挖掘算法研究财务造假的研究发展仍较为缓慢,\citep{LU}应用logistic模型、判别分析模型、神经网络模型等对123家上市公司财务造假样本与非财务造假样本进行分类预测,得出logisitc模型的预测效果较好,\citep{MingLi}应用主成分分析方法和CART算法建立财务风险预测模型,\citep{mixedcore}通过线性组合构造混合核函数,建立基于混合核学习的支持向量机财务欺诈预测模型,与单核的支持向量机模型相比,其模型的鲁棒性与识别精度都有所提高。

\section{研究评述}{Research Review}
国外财务报表造假的甄别的研究方式相对于国内更加成熟,但是建模的样本数据具有特定的西方国家市场经济环境,并没有具体结合中国的国情,存在一定的局限性,模型指标的构建缺乏通用性,因此模型的应用推广能力不足,在解决我国实际问题中预测效果并不理想。

机器学习算法的预测能力非常强,但是在解释性角度有一定的不足。

鉴于此,本文在国内外研究的基础上,对财务报表数据进行统计分析,从财务盈利能力、运营能力、资产管理效率等角度,并引入一些用于判断财务造假的账目数据。本文的主要目的在于探究中国上市财务造假最容易发生在哪些账目,以及如何尽量提高准确率以达到更好的财务监管的目的。