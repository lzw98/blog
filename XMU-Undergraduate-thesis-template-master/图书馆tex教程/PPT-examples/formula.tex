%# -*- coding:utf-8 -*-
% 导言区
\documentclass{ctexart}
\usepackage{amsmath,amssymb}


% 正文区(文稿区)
\begin{document}
    \section{简介}
    \LaTeX 将排版内容分为文本模式和数学模式。文本模式用于普通文本排版,数学模式用于数学公式排版。

    \section{行内公式}
    \subsection{美元符号}
    交换律是 $a+b=b+a$,如$1+2=2+1=3$。
    \subsection{小括号}
    交换律是 \(a+b=b+a\),如\(1+2=2+1=3\)。
    \subsection{math环境}
    \begin{math}a+b=b+a\end{math}

    \section{上下标}
    \subsection{上标}
    $3x^2-x+2=0$
    \subsection{下标}
    $a_1,a_2,...,a_100$

    \section{希腊字母}
    $\alpha$
    $\beta$
    $\gamma$
    $\epsilon$
    $\pi$
    $\omega$

    $\Gamma$
    $\Delta$
    $\Pi$
    $\Omega$

    $\alpha^2 + \beta^2 + \gamma = 0$
    \section{数学函数}
    $\log$
    $\sin$
    $\cos$
    $\arcsin$
    $\arccos$
    $\ln$

    $\cos^2x + \sin^2x = 1 $

    $\sqrt{2}$
    $\sqrt{x^2 + y^2}$
    $\sqrt{2+ \sqrt{2}}$
    $\sqrt[4]{x}$

    \section{分式}
    大于是体积的$3/4$

    大于是体积的 $\frac{3}{4}$

    \section{行间公式}
    \subsection{美元符号}
    交换律是 $$a+b=b+a$$ 如$$1+2=2+1=3$$
    \subsection{中括号}
     交换律是 \[a+b=b+a\] 如\[1+2=2+1=3\]
    \subsection{displaymath环境}
    \begin{displaymath} a+b=b+a \end{displaymath}
    \begin{displaymath} 1+2=2+1 \end{displaymath}
    \subsection{自动编号公式equation环境,单行单个公式}
        \begin{equation}
            a+b=b+a
        \end{equation}
        \begin{equation}
            1+2=2+1
        \end{equation}
        如公式\ref{eq:c}所示,如公式\ref{eq:t}所示
        \begin{equation}
            a+b=b+a
            \label{eq:c}
        \end{equation}
        \begin{equation}
            a+b=b+a
            \label{eq:t}
        \end{equation}
        equation只能生成单行单个公式,如\ref{eq:eq}所示。
        \begin{equation}
            \label{eq:eq}
            x^2+y^2=z^2 \\
            \sqrt{abc}-\sqrt[2]{edf} =\sum_2 z
        \end{equation}

    \subsection{不编号公式equation*环境}
        \begin{equation*}
            1+2=2+1
        \end{equation*}
    \section{多行公式}
    \subsection{gather环境,多行多编号}
        \begin{gather}
            a + b = b + a \\
            ab ba
        \end{gather}
    \subsection{gather*环境,多行无编号}
        \begin{gather*}
            a + b = b + a \\
            ab ba
        \end{gather*}
        %在\\前使用\notag,阻止编号
        \begin{gather}
            a + b = b + a \notag \\
            3^2 + 4^2 = 5^2 \notag \\
            a^2 + b^2 = c^2
        \end{gather}
    \subsection{align环境和align*环境}
        % 用&进行对齐
        \begin{align}
            x &= t+1 \\
            y+1 &= 2
        \end{align}
        % 不带编号,任意位置对齐
        \begin{align*}
            x &= t+1& x&=6+2 &x&=t+8 \\
            y &= 2 &y&=12 &y &=\cos x
        \end{align*}
    \subsection{split环境:对齐采用align环境的方式,编号在中间}
        \begin{equation}
        \begin{split}
            \label{eq:eq}
            x^2+y^2 &=z^2 \\
            \sqrt{abc}-\sqrt[2]{edf} &=\sum_2 z
        \end{split}
        \end{equation}
     \subsection{cases环境:类似分段函数的公式排版}
     \begin{equation}
        %每行公式中使用&分割为两部分
        %通常表示值和后面的条件
        D(x) =
        \begin{cases}
           1,& \text{如果} x \in \mathbb{Q};\\
           0,& \text{如果} x \in
           \mathbb{R}\setminus \mathbb{Q}.
        \end{cases}
        \end{equation}
        如果需要在公式中输入中文,那么可以用text命令临时切换到文本模式。

        公式的编号和交叉引用也是自动实现的,大家在排版中,要习惯于采用自动化的方式处理注入图、表、公式的编号和交叉引用。
\end{document}
