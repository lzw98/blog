%# -*- coding:utf-8 -*-
\pagenumbering{arabic}
\chapter{绪论}
\echapter{Introduction}
\section {研究背景和意义}
\esection{Research Background and Significances}
国内外有很多组织都已经开展了CRM的研究和开发工作,从管理到技术都有。在市场上,也开发了不少的CRM产品。虽然CRM产品具有各种不同规模,不同功能等特性,但是从其主要的功能特性上区分,CRM 产品可以分为以下几种
\subsection{国内外研究现状}
\esubsection{Research Status at Home and Abroad}
经过十几年金融电子化的改造,银行业经历了脱机业务处理,联机业务处理,数据集中处理。随着银行业网络化服务和管理信息化,业务虚拟化时代的来临。数据大集中告别手工的电子化后,为下一步进行信息化,利用数据仓库和管理系统把数据转化为有用的信息以支持管理决策,并最终实现知识化提供了信息基础\footnote{韩愈(768-824),字退之,河南河阳(
  今河南孟县)人,自称郡望昌黎,世称韩昌黎。幼孤贫刻苦好学,德宗贞元八年进士。}。数据大集中后实时业务处理系统较为先进和方便,但尚缺乏强大的业务支持管理系统,而这正是中国银行金融企业全面管理集中,全面成本控制,全面客户服务,全面风险防范和全面面向未来所必需的,如公式\ref{equ1}。
  \begin{equation}\label{equ1}
    \hat{a}+\hat{a}=\sqrt{{ab}^2}+\sqrt[n]{abc}
  \end{equation}

当今世界经济正朝着全球市场一体化,企业生存数字化,商业竞争国际化的方向发展,在这样的大背景下,以互联网,知识经济,高薪技术为代表,以满足客户的需求为核心的新经济迅速发展开来。在这新经济时代,企业的产品服务不再是竞争的核心,客户成为决定企业胜败的关键,这使得对客户关系管理的研究兴起,如表\ref{dataT}。
\renewcommand\arraystretch{1.5}
\begin{table}[!htb]
\centering
\caption{基本资料表}
\label{dataT}
\begin{tabular}{|l|l|l|l|l|}
\hline
\textbf{字段名称} & \textbf{字段类型} & \textbf{长度} & \textbf{字段描述} & \textbf{备注} \\ \hline
账户号           & Number        & 30          &               & 主键          \\ \hline
密码            & Number        & 30          & 加密            &             \\ \hline
姓名            & Varchar       & 50          &               &             \\ \hline
电子邮箱          & Varchar       & 50          & VIP客户必填       &             \\ \hline
\end{tabular}
\end{table}


银行服务的营销理念转成以客户为中心也有一个发展过程,在以前的时候,我国的商业银行,特别是国有商业银行,可以称为是“皇帝的女儿” ,各系统设置都是以账户为中心,为社会提供的银行服务都是有限的,品种单一的,对老百姓和机构都是一视同仁,无所谓“优质”服务与否,没有客户是否满意的概念,用来衡量银行经营目标和员工工作目标完成情况的重要指标\footnote{曾任监察御史,因上疏请免关中赋役,贬为阳山县令。后随宰相裴度平定淮西迁刑部侍郎,又因上表谏迎佛骨,贬潮州刺史。做过吏部侍郎,死谥文公,故世称韩吏部、韩文公。是唐代古文运动领袖,与柳宗元合称韩柳。诗力求险怪新奇,雄浑重气势。}。而如今日益激烈和全球化的竞争使银行逐渐认识到:银行的资产状况越来越取决于客户的状况,而要维系老客户,争取新客户以保持银行的竞争力,就必须融入CRM(Customer Relationship Management, 客户关系管理)的管理理念并建立相应的CRM系统,如公式\ref{equ2}。
\begin{align}\label{equ2}
  x&\equiv y+1(mod ~m^{2}) \\
  x&\equiv y+1mod ~m^{2} \nonumber \\
  x&\equiv y+1(m^{2}) \nonumber
\end{align}

最近10年来,全球化的竞争格局迫使商业银行不断扩展业务范围,而在技术水平相差不大的情况下,竞争也就聚焦在高效益的客户上。客户状况决定着银行的资产状况,具有良好发展前途的优质客户是银行优质资产的源泉,因此,客户是银行最重要的资源,银行对客户需求的满足能力是银行能否与客户保持紧密联系,获得发展的关键所在。在这样的情况下,越来越多的银行开始认识并引进

\section{国内外研究现状}
\esection{Research Status at Home and Abroad}
目前,国内外有很多组织都已经开展了CRM的研究和开发工作,从管理到技术都有。在市场上,也开发了不少的CRM产品。虽然CRM产品具有各种不同规模,不同功能等特性,但是从其主要的功能特性上区分,CRM产品\ucite{qzm2008}可以分为以下几种

1.	传统ERP型\ucite{djh2011}

传统ERP型CRM产品主要是许多主要的传统后端ERP厂商,如Oracle, SAP, PeopleSoft 等,进入CRM 市场所开发的CRM产品。该类产品的最大特点是采用了客户智能,融会贯通的交流渠道和基于Internet 技术的应用体系结构。其中客户智能是指该产品拥有强健的信息处理和商业分析能力,能够跨越多个软件模块和业务单位,对客户进行全面和智能的分析。融会贯通的交流渠道是指该产品将多种客户交流渠道集成起来,使各种渠道融会贯通,以保证企业,客户都能得到完整,准确和一致的信息。该类型的产品尤其以Oracle Application 为代表,如图\ref{f1}。
\begin{figure}[htb]
  \centering
  % Requires \usepackage{graphicx}
  \includegraphics[width=10cm]{1.jpg}\\
  \caption{图片1}\label{f1}
\end{figure}


2.	数据分析型\ucite{tfp}

数据分析型CRM产品认为数据仓库是进行客户分析的基础。数据仓库所建立的客户数据库使企业能收集到更详细的客户信息档案,以便对现有客户提供更好的服务,也可以建立一个预测模型,尽可能准确地预报客户流失的概率和可能性,以便及早采取相应的措施。利用数据挖掘工具和统计模型对数据仓库的数据进行研究,可以分析顾客的购买习惯,广告成功率和其他战略性的信息。这里恶性的产品是以Wal-Mart公司的数据仓库为代表。

\section{研究目的和研究内容}
\esection{Research aims and content}
本文的研究目的在于开发的CRM系统以及相关的数据仓库平台,使银行金融,信贷,决策,推广等部门更方便快捷准确有效的制定目标决策规划,并找出目标重点客户,进行按各个行业,各个产品,主要客户等各个层次的效益分析。业务部门依据各种报表,为客户提供一对一的贴心服务,同时为银行领导和金融部门的快速决策提供了良好的支持,提高了银行整体的竞争力,如图\ref{f2}。
\begin{figure}[htbp]
  \centering
  % Requires \usepackage{graphicx}
  \includegraphics[width=8cm]{2.jpg}\\
  \caption{图片2}\label{f2}
\end{figure}


本文意在实现一个CRM系统,该系统根据客户关系管理理论应具备的三大子系统,分别为营销管理子系统,销售管理子系统,服务管理子系统。主要实现以下几点:

1. 建设管理信息体系架构。通过技术手段,把现有的各种业务经营和服务渠道系统资源联系在一起,通过对信息的采集,提炼,建议起管理信息平台。

2. 为营销提供资料,使客户忠诚度得到提高。对客户进行评价,发现优质客户,重点发展该种客户,维持好客户的良好关系,提高客户重复交易行为。

3. 发展新客户.发展新的客户关系,了解不同客户的不同需求,提高客户满意度,进而发挥最大的促销能力,增加业务量,提高经济效益。

4. 减少成本。通过技术与业务流程的整合,精简业务流程,防范业务操作风险,节省成本。

5. 减少成本。通过技术与业务流程的整合,精简业务流程,防范业务操作风险,节省成本。

\section{本文组织结构}
\esection{Outline of the Dissertation}
全文共分为五章:

第1章是绪论,主要介绍了系统的研究目的和意义,以及国内外的研究现状。

第2章是相关技术介绍,对本系统实现所涉及到的一些框架和技术进行了简单的介绍。

第3章是系统需求分析,对系统进行概述,并对其需求进行分解,对系统的整个架构进行了分析。

第4章是系统设计,分析了系统的设计原则和目标,介绍了系统技术构架设计,并给出了系统的总体功能架构图,同时对系统项目管理设计也做了一定的介绍。

第5章是系统实现,对系统的界面、业务、工作流等方面做了实现功能的描述。列出了一些关键的相关代码。

第6章总结与展望部分,对本文进行了总结,并对下一步的工作进行了展望。





